\documentclass[a4paper, 11pt]{article}

\usepackage{kotex} % Comment this out if you are not using Hangul
\usepackage{fullpage}
\usepackage{hyperref}
\usepackage{amsthm}
\usepackage[numbers,sort&compress]{natbib}

\theoremstyle{definition}
\newtheorem{exercise}{Exercise}
\newtheorem{solution}{Expected Solutions}

\begin{document}
%%% Header starts
\noindent{\large\textbf{IS-521 Activity Proposal}\hfill
                \textbf{Your Name}} \\
         {\phantom{} \hfill \textbf{GitHub ID}} \\
         {\phantom{} \hfill Due Date: April 15, 2017} \\
%%% Header ends

\section{Activity Overview}

제가 생각하는 활동은 모듈화 프로그래밍과 Makefile을 작성하는 실습입니다.

프로그래밍할 때 모듈화를 잘 하는 능력은 다양하게 도움이 됩니다.
일단 개발속도를 빠르게 하며, 의존성을 감소시켜 유지보수를 쉽게 할 수 있도록 해줍니다.
그리고 잘 만들어진 모듈은 다른 프로그램에서 재사용 할 수동 있습니다.~\cite{modularprogramming}
모듈화 프로그래밍을 많이 연습하고 모듈화에 익숙해지게 되면 자신이 만들어야 할 프로그램을 요구사항에 맞게 잘 세분화 할 수 있게 됩니다.

Makefile은 소스코드를 자동으로 빌드하기 위해 반드시 필요한 파일입니다.
규모가 큰 프로젝트에서 일부의 파일만 수정됐을때 Makefile을 통해 수정된 파일과
관련된 파일만 리빌드 함으로써 빌드시간을 효과적으로 단축시킬 수 있습니다.~\cite{makefilewiki}
Makefile은 다양한 문법과 예약어들이 존재하므로 실습을 통해 공부를 할 필요가 있다고 생각합니다.

모듈화 프로그래밍 실습을 통해 만들어진 소스코드들을 Makefile로 빌드할 수 있게 연습함으로써 
프로그래밍 스킬을 높이고, 개발시간을 효과적으로 단축시킵니다.

\section{Exercises}

\begin{exercise}

  이 활동에서 학생들은 특정 요구사항을 만족하는 main 하나의 함수로 이루어진 프로젝트를 전달받습니다.
학생들은 이 프로그램을 리팩토링 하면서 코드를 소스코드별로, 함수별로 나누는 연습을 합니다.

\end{exercise}

\begin{exercise}

  이 활동에서 학생들은 리팩토링한 소스코드들을 한번에 빌드할 수 있도록 Makefile을 만듭니다.
혹은 소스코드가 20개정도 되는 프로젝트를 가져와서 해당 프로젝트의 Makefile만 제거한 후,
직접 해당 프로젝트를 빌드하기 위한 Makefile을 만들어봅니다.

\end{exercise}

\section{Expected Solutions}

\begin{solution}

서로 연관된 기능을 가진 코드들을 모아서 새로운 소스코드릉 만든다. 새로 만든 소스코드 안에서 중복되는 코드들을 함수로 선언해줌으로써 모듈화를 진행한다.

\end{solution}

\begin{solution}

일단 프로젝트가 여러개가 있다면 프로젝트의 우선순위를 정한다. 그다음 각 프로젝트의 소스코드들을 확인하며 의존성을 확인한다. 프로젝트별, 소스코드별 의존성을 확인하면서 Makefile을 만든다. Makefile에서 사용되는 예약어들을 이용하여 짧고 간결하게 Makefile을 만든다.

\end{solution}

\bibliography{references}
\bibliographystyle{plainnat}

\end{document}
