\documentclass[a4paper, 11pt]{article}

\usepackage{kotex} % Comment this out if you are not using Hangul
\usepackage{fullpage}
\usepackage{hyperref}
\usepackage{amsthm}
\usepackage[numbers,sort&compress]{natbib}

\theoremstyle{definition}
\newtheorem{exercise}{Exercise}
\newtheorem{solution}{Expected Solutions}

\begin{document}
%%% Header starts
\noindent{\large\textbf{IS-521 Activity Proposal}\hfill
                \textbf{김민근}} \\
         {\phantom{} \hfill \textbf{pr0v3rbs}} \\
         {\phantom{} \hfill Due Date: April 15, 2017} \\
%%% Header ends

\section{Activity Overview}

제가 생각하는 활동은 커널모듈 프로그래밍과 Makefile을 작성하는 실습입니다.

커널모듈은 리눅스에서 하드웨어레벨에 접근하기 위한 드라이버의 한 종류인데,
동적으로 로드, 언로드함으로써 유연하게 사용할 수 있는 리눅스 드라이버입니다.~\cite{loadablekernelmodule}
최근 IoT가 유행하면서 많은 리눅스기반의 임베디드 장비들이 만들어지고 있으며,
이러한 임베디드 장비들은 각각 서로다른 종류의 하드웨어로 구성되어 있습니다.
그래서 각 시스템에서 하드웨어를 제어하기 위한 커널모듈을 제조사에서 직접 만드는 경우가 많습니다.
이때 리눅스 커널모듈에서 취약점이 발생하면 유저레벨이 아닌 커널레벨이므로 따로 권한상승을 할 필요가 없기 때문에 위험도가 더욱 큽니다.
커널모듈을 직접 만들어보면서 커널모듈이 어떻게 동작하고, 어떻게 구성되는지 공부합니다.

Makefile은 소스코드를 자동으로 빌드하기 위해 반드시 필요한 파일입니다.
규모가 큰 프로젝트에서 일부의 파일만 수정됐을때 Makefile을 통해 수정된 파일과
관련된 파일만 리빌드 함으로써 빌드시간을 효과적으로 단축시킬 수 있습니다.~\cite{makefilewiki}
Makefile은 다양한 문법과 예약어들이 존재하므로 실습을 통해 공부를 할 필요가 있다고 생각합니다.

커널 모듈 프로그래밍 실습으로 만든 소스코드를 Makefile로 빌드할 수 있게 연습함으로써 
프로그래밍 스킬을 높이고, 개발시간을 효과적으로 단축시킵니다.

\section{Exercises}

\begin{exercise}

  이 활동에서 학생들은 커널모듈을 이용해 리눅스 rootkit을 만들어봅니다.

\end{exercise}

\begin{exercise}

  이 활동에서 학생들은 앞서 만든 커널모듈을 빌드하기 위한 Makefile을 만듭니다.

\end{exercise}

\section{Expected Solutions}

\begin{solution}

  커널모듈에서 systemcall table을 참조하여 몇몇 주요 systemcall 함수를 후킹해보고, 악의적인 행위를 하는 rootkit을 만든다.

\end{solution}

\begin{solution}

  소스코드 및 프로젝트들의 의존성을 확인한다. 프로젝트별, 소스코드별 의존성을 확인하면서 Makefile을 만든다. Makefile에서 사용되는 예약어들을 이용하여 짧고 간결하게 Makefile을 만든다.

\end{solution}

\bibliography{references}
\bibliographystyle{plainnat}

\end{document}
